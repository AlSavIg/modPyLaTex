\documentclass[12pt, a4paper]{article}%
% \large == 14 pt (with 12 pt document)
% see C:/Users/User/Documents/Рабочие документы/ITMO/LATEX/Samples/fonts.gif

\usepackage[utf8]{inputenc}%
\usepackage{lmodern}%
\usepackage{textcomp}%
\usepackage{lastpage}%
\usepackage{fontenc}%
\usepackage{tempora}% Нормальный Times доступен в XeLaTex компиляторе
% (можно задать любой сторонний шрифт через fontenc) - ДЛЯ ДИПЛОМА
\usepackage[english, russian]{babel}

\usepackage{newtxmath}

\usepackage{geometry}%
\geometry{head=20pt,margin=0.5in,bottom=0.6in,includeheadfoot=True}%
\usepackage{color}%
\usepackage{tabu}%
\usepackage{ragged2e}%
\usepackage{longtable}[=v4.13]%
\usepackage[table]{xcolor}%
\usepackage{graphicx}%
\usepackage{fancyhdr}%
\usepackage{tabularx}%

\usepackage{lipsum}

\usepackage{alltt}
\usepackage{datetime}
\usepackage{setspace}
\usepackage{tocbibind}
\usepackage{csquotes}
\usepackage[
backend=biber,
style=alphabetic,
sorting=ynt,
]{biblatex}
\usepackage{listings}

\newdateformat{yeardate}{\THEYEAR}

\begin{document}
	% Можно создать команду по генерации титульного листа с подстановкой моих параметров (имя преподавателя, дата сдачи,
	% возможно, факультет, для унификации стандарта по остальным кафедрам, номер и название ЛР)
	\begin{titlepage}
		\newpage
		\begin{tabular}{c c}%
			\begin{minipage}[l]{0.59\textwidth}%

				\flushleft
				\begin{small}%
					\begin{center}
						\textbf{Санкт-Петербургский национальный исследовательский
							университет информационных технологий, механики и оптики}%
					\end{center}
				\end{small}%

				\begin{small}%
					\begin{center}
						\setstretch{0}
						\textbf{УЧЕБНЫЙ ЦЕНТР ОБЩЕЙ ФИЗИКИ ФТФ}%
					\end{center}
				\end{small}%
			\end{minipage}%

			&\begin{minipage}[l]{0.30\textwidth}%
				\flushright
				\includegraphics[scale=0.85]{C:/Users/User/Documents/Рабочие документы/ITMO/LATEX/Samples/ITMO_logo.jpg}%
			\end{minipage}%
		\end{tabular}

		% horizontal line
		\begin{center}
			\par\noindent\rule{0.95\textwidth}{2pt}
		\end{center}

		\begin{tabular}{p{1em} p{22em} p{3em}}%
			&
			\begin{minipage}[r]{0.31\textwidth}%
				\setstretch{1.2}
				Группа: \textbf{P32201}
				\newline%
				%
				Студент: \textbf{Савчук А. И.}%
				\newline%
				%
				Преподаватель: \textbf{Боярский К.К.}%
			\end{minipage}

			&\begin{minipage}[r]{0.32\textwidth}%
					\setstretch{1.2}
					К работе допущен: \underline{\textbf{13.12.2022}\hspace{1.2em}}%
					\newline%
					%
					Работа выполнена: \underline{\textbf{05.01.2023}\hspace{1.3em}}%
					\newline%
					%
					Отчет принят: \hrulefill%
			\end{minipage}%
		\end{tabular}%

		\vspace{7em}

		\begin{center}
			%\textsc{\textbf{}}
			\Large \textbf{Рабочий протокол и отчет} \linebreak
			\textbf{по лабораторной работе № 3.0.0}
		\end{center}

		\vspace{0.02em}

		\begin{center}
			\large
			ИЗУЧЕНИЕ ЭЛЕКТРИЧЕСКИХ СИГНАЛОВ \linebreak
			С ПОМОЩЬЮ ЛАБОРАТОРНОГО ОСЦИЛЛОГРАФА
		\end{center}

		%\vspace{6em}
		%\begin{alltt}
		%	Научный руководитель
		%	д.ф.м.н., Andrey В.А.
		%	Рецензент
		%	к.ф.-м.н. Olegovich В.И.
		%\end{alltt}


		\vspace{\fill}

		\begin{center}
			Санкт-Петербург \yeardate \today
		\end{center}
	\end{titlepage}
\end{document}