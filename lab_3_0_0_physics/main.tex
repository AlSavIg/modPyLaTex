\documentclass[12pt, a4paper]{article}%
\usepackage{import}

\import{./preamble/}{preamble_pdflatex.tex} % import позволяет свободно использовать относительные адреса

\addbibresource{./bib/sample.bib}

\begin{document}

    \import{./pages/title}{title_page.tex}

    %\large

    \setcounter{page}{2}
    \tableofcontents

    \newpage
    \section{Экспериментальные данные}

    \subsection{Теория}
    \lipsum[2-4]

    \subsection{Практика}
    \lipsum[2-4]

    \section{Обработка экспериментальных данных}

    \subsection{Результаты}
    \lipsum[2-4]
    \cite{Collaboration2020}
    \cite{Reissel2016}

    \newpage
    \import{./pages/}{complex_report.tex}

    \newpage
    \import{./pages/}{SourceCodeInputExample.tex}

    \newpage
    \printbibliography[
    heading=bibintoc,
    ]
    \printbibliography[
    heading=subbibintoc,
    title={Subbibliography}
    ]
    % Библиографию лучше всего задавать через {}, а не "".
    % В последнем случае могут возникать проблемы с переносом текста
\end{document}